\documentclass{beamer}
\usepackage[utf8]{inputenc}
\usetheme{Szeged}
\usecolortheme{beaver}
\usepackage[english]{babel}
\usepackage{ulem}
\usepackage{xmpmulti}
%Information to be included in the title page:
\title{Malayalam Text Recognition System}
\author{Aarya R. Shankar
			\and Amrith M
				 \and Anand R
					\and Sarathchandran S}
\institute{Department of Computer Science\and College of Engineering,Trivandrum}
\date{2015-2019}
 
\begin{document}

\frame{\titlepage}


\begin{frame}
\frametitle{Table of Contents:-}
\begin{itemize}
\item Problem Definition \& Objectives
\item Problem Domain \& Relevance
\item Literature Review
\item Design of the Project
\item Dataset Updates
\item Model Trials
\item Further Updates
\item Planning of Project
\item Conclusion



\end{itemize}

\end{frame}

\begin{frame}{Problem Definition \& Objectives}
    Problem
    \begin{itemize}
         \item
         Character Recognition System for Malayalam doesn't exist. Hence digitizing official documents become difficult
    \end{itemize}
    Objectives
    \begin{itemize}
        \item To develop a character recognition system for Malayalam language with at least 90\% accuracy.
        \item Deploy the project to a simple application for demonstration.
        \item Completely open source the project (Existing ones aren't completely open source).
    \end{itemize}
     
\end{frame}

\begin{frame}{Problem Domain \& Relevance}
    Problem Domain
    \begin{itemize}
         \item
         Deep-Learning (with Convolutional Neural Networks) 
    \end{itemize}
    Relevance
    \begin{itemize}
        \item Can be used for developing an OCR, which is useful for Govt. purposes.
    
    \end{itemize}
     
\end{frame}

\begin{frame}{Literature Review}
    \textbf{Analyzed Papers}
    \begin{itemize}
         \item
         Malayalam Character Recognition using Discrete
Cosine Transform - Sasidas S R
        \item Offline Malayalam Character Recognition using - CUSAT Team
         \item Lekha OCR - SPACE Kerala
    \end{itemize}
    
     
\end{frame}

\begin{frame}{Design of the Project: Dataset Updates}

    \begin{columns}[c] % the "c" option specifies center vertical alignment
    \column{\textwidth} % column designated by a command
    \begin{itemize}
        \item Received a non-uniformly distributed dataset from SPACE
        \item Cleaned up the dataset.

        \item Changed all images to 32X32 pixels

        \item Converted all Images to Grayscale

    \end{itemize}
     
    \end{columns}
\end{frame}



\begin{frame}{Design of the Project: Model Trials}
     \begin{itemize}
        \item Created a Test Model with the following features: 
        \item 2 Conv-Conv-Pool Layers (Intuition from VGG16)

        \item 2 Fully Connected Layers

        \item Output layer with Softmax Log Probabilities(133 Classes)
        
        \item Tested on a Sample dataset with 40\% accuracy (BAD)


    \end{itemize}
\end{frame}

\begin{frame}{Design of the Project: Further Updates}
     \begin{itemize}
        \item Getting More Data 
        
        \item Use of augmentation and ensembles

        \item Optimizing the model and hyperparameters


    \end{itemize}
\end{frame}

\begin{frame}{Planning of the Project}
\begin{itemize}
    \item Oct 25 - Final Data Setup in standard MNIST format
    \item Nov 10 - Optimizing the model (Layer,Parameters etc)
    \item Nov 20 - Completing the OCR and using trained weights for prediction. Implementing this to a system. 
\end{itemize}
\end{frame}

\begin{frame}
\frametitle{Conclusion}
More than half of the works are done. Little tweak on the dataset and the model is left for the project to end.
\end{frame}
\end{document}


